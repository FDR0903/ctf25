\documentclass[11pt,a4,table]{article}

\renewcommand{\footnotesize}{\fontsize{9pt}{11pt}\selectfont}

%%%%%%%%%%%%%%%%%%%%%%%%%%%%%%%%%%%%%%%%%%%%%%%%%%%%

% Page Dimensions
\usepackage[top=1.0in, bottom=1.0in, left= 0.80in, right=0.80in]{geometry}

%%%%%%%%%%%%%%%%%%%%%%%%%%%%%%%%%%%%%%%%%%%%%%%%%%%%

% Caption options
\usepackage[margin= 16pt , footnotesize ,  sc , labelfont=bf , labelsep=period , format=plain , justification=centerfirst]{caption}


%%%%%%%%%%%%%%%%%%%%%%%%%%%%%%%%%%%%%%%%%%%%%%%%%%%%
%load minimum set of other required packages
\usepackage{amsfonts}
\usepackage{amsmath}
\usepackage{amssymb}
\usepackage{amsthm}
\usepackage{array}
\usepackage{bbm}
\usepackage{booktabs,longtable}
\usepackage{eurosym}
\usepackage{floatrow}
\usepackage{graphicx}
\usepackage{hyperref}
\usepackage{harvard}
\usepackage[english]{babel}
\usepackage{multirow}
\usepackage{lscape}
\usepackage{pdflscape}
\usepackage{rotating}
\usepackage{subfig}
\usepackage{setspace}
\usepackage{titlesec}
\usepackage{verbatim}
\usepackage{xcolor}
\usepackage[T1]{fontenc}

\usepackage[utf8]{inputenc}


%%%%%%%%%%%%%%%%%%%%%%%%%%%%%%%%%%%%%%%%%%%%%%%%%%%%
%%%%%%%%%%%%%%%%%%%%%%%%%%%%%%%%%%%%%%%%%%%%%%%%%%%%%

% and tikz
\usepackage{tikz}
\usetikzlibrary{decorations.pathreplacing}
	\definecolor{cinnabar}{rgb}{0.89, 0.26, 0.2}
	\definecolor{ceruleanblue}{rgb}{0.16, 0.32, 0.75}

% and optimally spaces tables

\usepackage{tabularx, booktabs}
\newcolumntype{Y}{>{\centering\arraybackslash}X}


%%%%%%%%%%%%%%%%%%%%%%%%%%%%%%%%%%%%%%%%%%%%%%%%%%%%

%% Roman numerals command thing.
\makeatletter
\newcommand{\rmnum}[1]{\romannumeral #1}
\newcommand{\Rmnum}[1]{\expandafter\@slowromancap\romannumeral #1@}
\makeatother

%%%%%%%%%%%%%%%%%%%%%%%%%%%%%%%%%%%%%%%%%%%%%%%%%%%%

% fake section command
\newcommand{\fakesection}[1]{%
  \par\refstepcounter{section}% Increase section counter
  \sectionmark{#1}% Add section mark (header)
  \addcontentsline{toc}{section}{\protect\numberline{\thesection}#1}% Add section to ToC
  % Add more content here, if needed.
}


%%%%%%%%%%%%%%%%%%%%%%%%%%%%%%%%%%%%%%%%%%%%%%%%%%%%

%% ------------------------------------------
%% Journal of Finance conventions
%% ------------------------------------------

\renewcommand{\thesection}{\Roman{section}}
\renewcommand{\thesubsection}{\Alph{subsection}}
\renewcommand{\thetable}{\Roman{table}}
\renewcommand{\abstractname}{\textsc{ABSTRACT}}
\makeatletter
\renewcommand{\baselinestretch}{1.10}
\renewcommand{\topfraction}{0.99}
\renewcommand{\bottomfraction}{0.99}
\setcounter{topnumber}{2}
\setcounter{bottomnumber}{2}
\setcounter{totalnumber}{4}
\setcounter{dbltopnumber}{2}
\renewcommand{\dbltopfraction}{0.9}
\renewcommand{\textfraction}{0.07}
\renewcommand{\floatpagefraction}{0.9}
\renewcommand{\dblfloatpagefraction}{0.9}
\titleformat{\section}{\centering\large\bfseries}{\thesection.}{1em}{}
\titleformat{\subsection}{\normalsize\itshape}{\thesubsection.}{1em}{}
\titleformat{\subsubsection}{\normalsize\itshape}{\thesubsubsection.}{1em}{}


%%%%%%%%%%%%%%%%%%%%%%%%%%%%

% math shocks cuts, theorems, lemmas etc.
\newcommand{\inner}[2]{\left < #1 \big | #2 \right >}
\newcommand{\proj}[2]{\mbox{proj} \left ( #1 \big | #2 \right )}
\newcommand{\cov}[2]{\mbox{cov}\left (#1 ,#2 \right )}
\newcommand{\var}[1]{\mbox{var}[#1]}
\newcommand{\Gray}[1]{{\color{gray}#1}}
\newcommand{\Red}[1]{{\color{red}#1}}


\newtheorem{theorem}{Theorem}
\newtheorem{acknowledgement}{Acknowledgement}
\newtheorem{algorithm}{Algorithm}
\newtheorem{axiom}{Axiom}
\newtheorem{case}{Case}
\newtheorem{claim}{Claim}
\newtheorem{conclusion}{Conclusion}
\newtheorem{condition}{Condition}
\newtheorem{conjecture}{Conjecture}
\newtheorem{corollary}{Corollary}
\newtheorem{criterion}{Criterion}
\newtheorem{definition}{Definition}
\newtheorem{example}{Example}
\newtheorem{exercise}{Exercise}
\newtheorem{lemma}{Lemma}
\newtheorem{notation}{Notation}
\newtheorem{problem}{Problem}
\newtheorem{proposition}{Proposition}
\newtheorem{hypothesis}{Hypothesis}
\newtheorem{remark}{Remark}
\newtheorem{solution}{Solution}
\newtheorem{summary}{Summary}

%%%%%%%%%%%%%%%%%%%%%%%%%%%%%%%%%%%%%%%%%%%%%%%%%%%%

% and choose line spacing (comes from setspace)

%\singlespacing
%\onehalfspacing
%\doublespacing
%\setstretch{1.5}
%\setstretch{2}

% make quote right flushed.
\makeatletter{}
\g@addto@macro\quote\flushright
\makeatother


%%%%%%%%%%%%%%%%%%%%%%%%%%%%%%%%%%%%%%%%%%%%%%%%%%%%%%%%%%%%%%%%%%%%%%
%%%%%%%%%%%%%%%%%%%%%%%%%%%%%%%%%%%%%%%%%%%%%%%%%%%%%%%%%%%%%%%%%%%%%%


\begin{document}

\vspace*{-0.7in}

\begin{center}
 \textbf{Continuous Time Finance: Problem Set 1} \\
\textit{Binomial Model and Option Pricing} \\
Fayçal Drissi \\
\end{center} 


\begin{enumerate}
    \item Explain briefly what is 
    \begin{enumerate} 
    \item Sample space 
    \item Filtration 
    \item Probability measure and probability space
    \item Stochastic process adapted to a filtration
    \item and what is a martingale
	\end{enumerate}
    \item 
    Let us now assume $r=0,u=1/d=2$ and $S_0=4$ in a binomial setup
    \begin{enumerate}
    
        \item Determine the range of the real valued function $S_3:\Omega\rightarrow\mathbb{R}$, i.e. the set of all possible stock prices at the end of period 3.
    
        \item Determine the range of the real valued function $V_3:\Omega\rightarrow\mathbb{R}$, of a call option with exercise price $K=4$ and maturity $T=3$.
    \end{enumerate}
    
    Remark: functions like $S_3$ and $V_3$ are called random variables.
    
  
    \item In a two periods binomial model with $r=0, u=1/d=2$ and $S_0 = 4$:
    
    \begin{enumerate}
        \item Compute the price of a European call option with strike price $K=4$ and maturity $T=1$.
        
        \item For a European call option with strike price $K=4$ and maturity $T=2$ compute the corresponding hedging strategy by recursively computing $\Delta_1$ and $V_1$ and then $\Delta_0$ and $V_0$.
    \end{enumerate}
    
    \item A married put option consists of buying shares of a stock and simultaneously buying a put option with strike $K$ for the same number of shares (here the number is one). 
    \begin{enumerate}
        \item Write and plot the payoff of the option as a function of the terminal stock price.
        \item Describe the financial motivation to set such a strategy as an investor.
        \item What is the price of the option in the two-period binomial model ?
    \end{enumerate}
    
\end{enumerate}


\end{document}







