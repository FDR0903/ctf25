\documentclass[11pt,a4,table]{article}

\renewcommand{\footnotesize}{\fontsize{9pt}{11pt}\selectfont}

%%%%%%%%%%%%%%%%%%%%%%%%%%%%%%%%%%%%%%%%%%%%%%%%%%%%

% Page Dimensions
\usepackage[top=1.0in, bottom=1.0in, left= 0.80in, right=0.80in]{geometry}

%%%%%%%%%%%%%%%%%%%%%%%%%%%%%%%%%%%%%%%%%%%%%%%%%%%%

% Caption options
\usepackage[margin= 16pt , footnotesize ,  sc , labelfont=bf , labelsep=period , format=plain , justification=centerfirst]{caption}


%%%%%%%%%%%%%%%%%%%%%%%%%%%%%%%%%%%%%%%%%%%%%%%%%%%%
%load minimum set of other required packages
\usepackage{amsfonts}
\usepackage{amsmath}
\usepackage{amssymb}
\usepackage{amsthm}
\usepackage{array}
\usepackage{bbm}
\usepackage{booktabs,longtable}
\usepackage{eurosym}
\usepackage{floatrow}
\usepackage{graphicx}
\usepackage{hyperref}
\usepackage{harvard}
\usepackage[english]{babel}
\usepackage{multirow}
\usepackage{lscape}
\usepackage{pdflscape}
\usepackage{rotating}
\usepackage{subfig}
\usepackage{setspace}
\usepackage{titlesec}
\usepackage{verbatim}
\usepackage{xcolor}
\usepackage[T1]{fontenc}

\usepackage[utf8]{inputenc}


%%%%%%%%%%%%%%%%%%%%%%%%%%%%%%%%%%%%%%%%%%%%%%%%%%%%
%%%%%%%%%%%%%%%%%%%%%%%%%%%%%%%%%%%%%%%%%%%%%%%%%%%%%

% and tikz
\usepackage{tikz}
\usetikzlibrary{decorations.pathreplacing}
	\definecolor{cinnabar}{rgb}{0.89, 0.26, 0.2}
	\definecolor{ceruleanblue}{rgb}{0.16, 0.32, 0.75}

% and optimally spaces tables

\usepackage{tabularx, booktabs}
\newcolumntype{Y}{>{\centering\arraybackslash}X}


%%%%%%%%%%%%%%%%%%%%%%%%%%%%%%%%%%%%%%%%%%%%%%%%%%%%

%% Roman numerals command thing.
\makeatletter
\newcommand{\rmnum}[1]{\romannumeral #1}
\newcommand{\Rmnum}[1]{\expandafter\@slowromancap\romannumeral #1@}
\makeatother

%%%%%%%%%%%%%%%%%%%%%%%%%%%%%%%%%%%%%%%%%%%%%%%%%%%%

% fake section command
\newcommand{\fakesection}[1]{%
  \par\refstepcounter{section}% Increase section counter
  \sectionmark{#1}% Add section mark (header)
  \addcontentsline{toc}{section}{\protect\numberline{\thesection}#1}% Add section to ToC
  % Add more content here, if needed.
}


%%%%%%%%%%%%%%%%%%%%%%%%%%%%%%%%%%%%%%%%%%%%%%%%%%%%

%% ------------------------------------------
%% Journal of Finance conventions
%% ------------------------------------------

\renewcommand{\thesection}{\Roman{section}}
\renewcommand{\thesubsection}{\Alph{subsection}}
\renewcommand{\thetable}{\Roman{table}}
\renewcommand{\abstractname}{\textsc{ABSTRACT}}
\makeatletter
\renewcommand{\baselinestretch}{1.10}
\renewcommand{\topfraction}{0.99}
\renewcommand{\bottomfraction}{0.99}
\setcounter{topnumber}{2}
\setcounter{bottomnumber}{2}
\setcounter{totalnumber}{4}
\setcounter{dbltopnumber}{2}
\renewcommand{\dbltopfraction}{0.9}
\renewcommand{\textfraction}{0.07}
\renewcommand{\floatpagefraction}{0.9}
\renewcommand{\dblfloatpagefraction}{0.9}
\titleformat{\section}{\centering\large\bfseries}{\thesection.}{1em}{}
\titleformat{\subsection}{\normalsize\itshape}{\thesubsection.}{1em}{}
\titleformat{\subsubsection}{\normalsize\itshape}{\thesubsubsection.}{1em}{}


%%%%%%%%%%%%%%%%%%%%%%%%%%%%

% math shocks cuts, theorems, lemmas etc.
\newcommand{\inner}[2]{\left < #1 \big | #2 \right >}
\newcommand{\proj}[2]{\mbox{proj} \left ( #1 \big | #2 \right )}
\newcommand{\cov}[2]{\mbox{cov}\left (#1 ,#2 \right )}
\newcommand{\var}[1]{\mbox{var}[#1]}
\newcommand{\Gray}[1]{{\color{gray}#1}}
\newcommand{\Red}[1]{{\color{red}#1}}


\newtheorem{theorem}{Theorem}
\newtheorem{acknowledgement}{Acknowledgement}
\newtheorem{algorithm}{Algorithm}
\newtheorem{axiom}{Axiom}
\newtheorem{case}{Case}
\newtheorem{claim}{Claim}
\newtheorem{conclusion}{Conclusion}
\newtheorem{condition}{Condition}
\newtheorem{conjecture}{Conjecture}
\newtheorem{corollary}{Corollary}
\newtheorem{criterion}{Criterion}
\newtheorem{definition}{Definition}
\newtheorem{example}{Example}
\newtheorem{exercise}{Exercise}
\newtheorem{lemma}{Lemma}
\newtheorem{notation}{Notation}
\newtheorem{problem}{Problem}
\newtheorem{proposition}{Proposition}
\newtheorem{hypothesis}{Hypothesis}
\newtheorem{remark}{Remark}
\newtheorem{solution}{Solution}
\newtheorem{summary}{Summary}

%%%%%%%%%%%%%%%%%%%%%%%%%%%%%%%%%%%%%%%%%%%%%%%%%%%%

% and choose line spacing (comes from setspace)

%\singlespacing
%\onehalfspacing
%\doublespacing
%\setstretch{1.5}
%\setstretch{2}

% make quote right flushed.
\makeatletter{}
\g@addto@macro\quote\flushright
\makeatother


%%%%%%%%%%%%%%%%%%%%%%%%%%%%%%%%%%%%%%%%%%%%%%%%%%%%%%%%%%%%%%%%%%%%%%
%%%%%%%%%%%%%%%%%%%%%%%%%%%%%%%%%%%%%%%%%%%%%%%%%%%%%%%%%%%%%%%%%%%%%%


\begin{document}

\vspace*{-0.7in}

\begin{center}
 \textbf{Continuous Time Finance: Problem Set 4} \\
 \textit{Risk Neutral Pricing and Connection with PDEs } \\
Ilaria Piatti
\end{center} 


\begin{enumerate}
    \item \textbf{Heston Model} \\
    Suppose that under the risk-neutral measure $\Tilde{P}$, a stock price follows:
    \begin{equation}
        dS(t) = rS(t) dt + \sqrt{V(t)} S(t) d\Tilde{W}_1(t),
    \end{equation}
    where $r$ is constant and the volatility $\sqrt{V(t)}$ is itself a stochastic process:
    \begin{equation}
        dV(t) = (a - bV(t)) dt + \sigma \sqrt{V(t)} d\Tilde{W}_2(t),
    \end{equation}
    where $a,b$ and $\sigma$ are positive constants, and $\Tilde{W}_1(t)$ and $\Tilde{W}_2(t)$ are correlated Brownian motions under $\Tilde{P}$, with
    \begin{equation*}
        d\Tilde{W}_1(t) d\Tilde{W}_2(t) = \rho dt,
    \end{equation*}
    for some $\rho\in(-1,1)$. The two-dimensional process $(S(t), V(t))$, governed by the SDEs (1) and (2), is a Markov process.\\
    The risk-neutral price of a call expiring at time $T>t$ is:
    \begin{equation*}
        c(t,S(t), V(t)) = \Tilde{E}\left[e^{-r(T-t)}(S(T)-K)^+|\mathcal{F}_t\right].
    \end{equation*}
    This problem shows that the function $c(t,s,v)$ satisfies the PDE:
    \begin{equation}
        c_t + rsc_s + (a-bv)c_v + \frac{1}{2}s^2 vc_{ss} + \rho \sigma svc_{sv} + \frac{1}{2}\sigma^2 vc_{vv} = rc,
    \end{equation}
    with boundary condition
    \begin{equation}
        c(T,s,v) = (s-K)^+ \textit{ for all } s\geq 0, v\geq 0. 
    \end{equation}
    
    \begin{enumerate}
        \item Show that $e^{-rt}c(t,S(t),V(t))$ is a martingale under $\Tilde{P}$ and use this fact to obtain the PDE (3).
    
        \item  Suppose there are functions $f(t,x,v)$ and $g(t,x,v)$ satisfying
        \begin{align}
            f_t + \left(r + \frac{1}{2}v\right) f_x + (a - bv + \rho \sigma v) f_v + \frac{1}{2}vf_{xx} + \rho\sigma v f_{xv} + \frac{1}{2}\sigma^2 v f_{vv} &= 0,\\
            g_t + \left(r - \frac{1}{2}v\right) g_x + (a - bv) g_v + \frac{1}{2}vg_{xx} + \rho \sigma v g_{xv} + \frac{1}{2}\sigma^2 vg_{vv} &=0.
        \end{align}
        Show that if we define
        \begin{equation}
            c(t,s,v) = sf(t,\log s, v) - e^{-r(T-t)}Kg(t,\log s, v),
        \end{equation}
        then $c(t,s,v)$ satisfies the PDE (3).
        \newpage
        
        \item Suppose a pair of processes $(X(t), V(t))$ is governed by the SDEs:
        \begin{align}
            dX(t) &= \left(r + \frac{1}{2}V(t) \right) dt + \sqrt{V(t)} dW_1(t),\\
            dV(t) &= (a - bV(t) + \rho \sigma V(t)) dt + \sigma\sqrt{V(t)} dW_2(t),
        \end{align}
        where $W_1(t)$ and $W_2(t)$ are Brownian motions under some probability measure $P$ with \\$dW_1(t)dW_2(t) = \rho dt$. Define
        \begin{equation*}
            f(t,x,v)=E^{t,x,v}\left[\mathbbm{1}_{\{X(T)\geq \log K\}}\right].
        \end{equation*}
        Show that $f(t,x,v)$ satisfies the PDE (5) and the boundary condition
        \begin{equation*}
            f(T,x,v) = \mathbbm{1}_{\{x\geq\log K\}}, \textit{ for all } x\in \mathbb{R} \textit{ and } v\geq 0.
        \end{equation*}
        
        \item Suppose a pair of processes $(X(t), V(t))$ is governed by the SDEs:
        \begin{align}
            dX(t) &= \left( r - \frac{1}{2}V(t) \right) dt + \sqrt{V(t)} dW_1(t),\\
            dV(t) &= (a - bV(t)) dt + \sigma\sqrt{V(t)} dW_2(t),
        \end{align}
        where $W_1(t)$ and $W_2(t)$ are Brownian motions under some probability measure $P$ with \\$dW_1(t) dW_2(t) = \rho dt$. Define
        \begin{equation*}
            g(t,x,v) = E^{t,x,v}\left[\mathbbm{1}_{\{X(T)\geq\log K\}}\right].
        \end{equation*}
        Show that $g(t,x,v)$ satisfies the PDE (6) and the boundary condition
        \begin{equation*}
            g(T,x,v) = \mathbbm{1}_{\{x\geq \log K\}}, \textit{ for all } x\in\mathbb{R} \textit{ and } v\geq 0.
        \end{equation*}
        
        \item Show that with $f(t,x,v)$ and $g(t,x,v)$ as in (c) and (d), the function $c(t,x,v)$ of (7) satisfies the boundary condition (4).
    \end{enumerate}
    
    
    \item \textbf{Option to exchange an asset for another}\\
    We consider a financial market in which there are two risky assets with prices $S_t^1$ and $S_t^2$ at time $t$ and a riskless asset $S_t^0=e^{rt}$. The stock prices follow the SDEs:
    \begin{align*}
        dS_t^1 &= S_t^1(\mu_1 dt + \sigma_1 dB_t^1),\\
        dS_t^2 &= S_t^2(\mu_2 dt + \sigma_2 dB_t^2),
    \end{align*}
    where $B_t^1$ and $B_t^2$ are two independent Brownian motions defined on a probability space $(\Omega,\mathcal{G},P)$, and denote by $\mathcal{G}_t$ the sigma algebra generated by the Brownian motions up to time $t$.\\
    We study the pricing and hedging of an option giving the right to exchange one of the risky assets for the other at time $T$.
    
    \begin{enumerate}
        \item We set
        \begin{equation*}
            \theta_1=\frac{\mu_1 - r}{\sigma_1} \textit{ and } \theta_2=\frac{\mu_2 - r}{\sigma_2}.
        \end{equation*}
        Show that the process defined by
        \begin{equation*}
            M_t=e^{-\theta_1 B_t^1 -\theta_2 B_t^2 -\frac{1}{2}(\theta_1^2+\theta_2^2)t},
        \end{equation*}
        is a martingale with respect to the filtration $(\mathcal{G}_t)_{t\in[0, T]}$.
        
        \item Let $\Tilde{P}$ be the probability with Radon-Nikodym derivative $M_t$ with respect to $P$, i.e.
        \begin{equation*}
            \frac{d\Tilde{P}}{dP} = M_t
        \end{equation*}
        We introduce the processes
        \begin{equation*}
            W_t^1 = B_t^1 + \theta_1 t \textit{ and } W_t^2 = B_t^2 + \theta_2 t.
        \end{equation*}
        Derive, under the probability $\Tilde{P}$, the joint moment generating function of $W_t=(W_t^1, W_t^2)$:
        \begin{equation*}
            \Tilde{E}\left[e^{uW_t}\right].
        \end{equation*}
        Then deduce that, for any $t\in[0,T]$, the random variables $W_t^1$ and $W_t^2$ are independent normal variables with mean zero and variance $t$ under $\Tilde{P}$.
        
        \item Show that, under $\Tilde{P}$, the discounted prices
        \begin{equation*}
            \Tilde{S}_t^1=e^{-rt}S_t^1 \textit{ and } \Tilde{S}_t^2 = e^{-rt} S_t^2
        \end{equation*}
        are martingales.
    \end{enumerate}
    
    We want to price and hedge a European option, with maturity $T$, giving to the holder the right to exchange one unit of the asset 2 for one unit of the asset 1. From his initial wealth, the premium, the writer of the option builds a strategy, defining at any time $t$ a portfolio made of $H_t^0$ units of the riskless asset and $H_t^1$ and $H_t^2$ units of the assets 1 and 2, respectively, in order to generate, at time $T$, a wealth equal to $(S_T^1-S_T^2)^+$. A trading strategy will be defined by the three adapted processes $H^0, H^1$ and $H^2$.
    \begin{enumerate}
        \setcounter{enumii}{3}
        
        \item Define precisely the self-financing strategy and prove that, if
        \begin{equation*}
            \Tilde{V}_t = e^{-rt} V_t
        \end{equation*}
        is the discounted value of the strategy, we have:
        \begin{equation*}
            d\Tilde{V}_t = H_t^1 e^{-rt} S_t^1\sigma_1 dW_t^1 + H_t^2 e^{-rt} S_t^2 \sigma_2 dW_t^2.
        \end{equation*}
        
        \item Prove that if a self-financing strategy has terminal value equal to $V_T=(S_T^1-S_T^2)^+$, then its value at any time $t<T$ is given by $V_t=F(t,S_t^1,S_t^2)$, where the function $F$ is defined by:
        \begin{equation}
            F(t,x_1,x_2) = \Tilde{E}\left(x_1e^{\sigma_1(W_T^1-W_t^1)-\frac{\sigma_1^2}{2}(T-t)}-x_2e^{\sigma_2(W_T^2W_t^2)-\frac{\sigma_2^2}{2}(T-t)}\right)^+.
        \end{equation}
        
        \item Find a parity relationship between the value of the option with payoff $(S_T^1-S_T^2)^+$ and the symmetrical option $(S_T^2-S_T^1)^+$, similar to the put-call parity relationship and give an example of arbitrage opportunity when this relation does not hold.
        
        \item We set $\Tilde{C}_t = e^{-rt}F(t,S_t^1, S_t^2)$. Apply Itô's formula to $\Tilde{C}_t$ and then exploit the fact that $\Tilde{C}_t$ is a martingale to derive a PDE for the function $F$.
        
        \item Comparing the dynamics of the discounted option price with the discounted value of the self-financing strategy, derive the positions in the two stocks, $H_t^1$ and $H_t^2$, in the hedging portfolio.
    \end{enumerate}
    
    
\end{enumerate}


\end{document}







