\documentclass[handout,8pt]{beamer}

\usetheme{CambridgeUS}
%\setbeameroption{show notes}

\usepackage{tabularx,color,colortbl}
\usepackage{amsmath,amsfonts,amssymb}


%\usepackage[comma]{natbib}
%\usepackage{theorem}
\usepackage{pdfsync}
\usepackage{graphicx}
\usepackage{epstopdf}
%\usepackage{caption}
\usepackage{tikz}
\usetikzlibrary{patterns}
%\usetikzlibrary{matrix}
%\usetikzlibrary{arrows,backgrounds,decorations}

%\usepackage{pgfpages}
%\pgfpageuselayout{resize to}[a4paper,border,shrink=5mm,landscape]

\setbeameroption{show notes} %un-comment to see the notes
\setbeamertemplate{note page}[plain]

\makeatother
\setbeamertemplate{footline}
{
  \leavevmode%
  \hbox{%
  \begin{beamercolorbox}[wd=.4\paperwidth,ht=2.25ex,dp=1ex,center]{author in head/foot}%
    \usebeamerfont{author in head/foot}\insertshortauthor
  \end{beamercolorbox}%
  \begin{beamercolorbox}[wd=.6\paperwidth,ht=2.25ex,dp=1ex,center]{title in head/foot}%
    \usebeamerfont{title in head/foot}\insertshorttitle\hspace*{3em}
    \insertframenumber{} / \inserttotalframenumber\hspace*{1ex}
  \end{beamercolorbox}}%
  \vskip0pt%
}
\makeatletter
\setbeamertemplate{navigation symbols}{}


\newcommand{\E}{\mathbb{E}}
\newcommand{\var}{\mathrm{Var}}
\newcommand{\p}{\mathbb{P}}
\newcommand{\f}{\mathcal{F}}
\newcommand{\cov}{\mathrm{Cov}}
\newcommand{\corr}{\mathrm{Corr}}
\newcommand{\R}{\mathbb{R}}
\newcommand{\N}{\mathcal{N}}
\newcommand{\USD}{\text{USD}}
\newcommand{\HC}{\text{HC}}
\newcommand{\FC}{\text{HC}}

\newcommand{\textred}{\textcolor{red}}
\newcolumntype{H}{>{\columncolor[rgb]{1.00,1.00,0.00}}c}

\setbeamertemplate{footline}[frame number]


% itemize options
\let\OLDitemize\itemize
\renewcommand\itemize{\OLDitemize\addtolength{\itemsep}{10pt}}

\setbeamertemplate{itemize/enumerate body begin}{\large}
\setbeamertemplate{itemize/enumerate subbody begin}{\large}

\makeatletter
\define@key{beamerframe}{wide}[30pt]{%
  \def\beamer@cramped{\itemsep #1\topsep0.5pt\relax}}
\makeatother


%%%%%%%%%%%%%%%%%%%%%%%%%%%%%%%%%%%%%%%%%%%%%%%%%%%%
%%%%%%%%%%%%%%%%%%%%%%%%%%%%%%%%%%%%%%%%%%%%%%%%%%%%%

% and tikz
\usepackage{tikz}
\usetikzlibrary{decorations.pathreplacing}
	\definecolor{cinnabar}{rgb}{0.89, 0.26, 0.2}
	\definecolor{ceruleanblue}{rgb}{0.16, 0.32, 0.75}

\usetikzlibrary{patterns}
\usetikzlibrary{matrix}
\usetikzlibrary{arrows,backgrounds,decorations}

\newcommand*\circled[1]{\tikz[baseline=(char.base)]{
            \node[shape=circle,draw,inner sep=2pt] (char) {#1};}}


% ------------------------------------------------------------------------------------------------------------------------------------------------------------------------- %
% ------------------------------------------------------------------------------------------------------------------------------------------------------------------------- %
% ------------------------------------------------------------------------------------------------------------------------------------------------------------------------- %




\title{Continuous Time Finance: Solution Exercise Set 3}
\author{\textbf{Paul Whelan}}
\institute{Copenhagen Business School}
\date{2021}

\begin{document}

\begin{frame}
     \titlepage
\end{frame}


% ------------------------------------------------------------------------------------------------------------------------------------------------------------------------- %
% ------------------------------------------------------------------------------------------------------------------------------------------------------------------------- %
% ------------------------------------------------------------------------------------------------------------------------------------------------------------------------- %




\begin{frame}[allowframebreaks]{ }



\begin{enumerate}
    \item Let $S_t=S_0\exp\{\sigma W_t+(\alpha - \frac{1}{2}\sigma^2)t\}$ be a geometric Brownian motion and let $p$ be a positive constant. \newline
    
    Compute $d(S_t^p)$, the differential of $S_t$ raised to the power of $p$.\newline
    
    Without loss of generality, we assume $p \neq 1 $. \newline
    
    Since $(x^p)'=px^{p-1}$ and $(x^p)''=p(p-1)x^{p-2}$, we have:
    \begin{align*}
        d(S_t^p) &= pS_t^{p-1} dS_t + \frac{1}{2}p(p-1)S_t^{p-2} (dS_t)^2\\
        &= pS_t^{p-1}(\alpha S_t dt + \sigma S_t dW_t) +\frac{1}{2}p(p-1)S_t^{p-2}\sigma^2 S_t^2 dt\\
        &= pS_t^p(\alpha dt + \sigma dW_t) + \frac{1}{2}p(p-1)S_t^p\sigma^2 dt\\
        &= S_t^p \left[p (\alpha + \frac{1}{2}(p-1)\sigma^2) dt + p \sigma dW_t \right]
    \end{align*}
    using the fact that
    \begin{equation*}
        dS_t = \alpha S_t dt + \sigma S_t dW_t.
    \end{equation*}
    Therefore, $S_t^p$ is still a geometric Brownian motion.
    
    \item Let $W_t$ be a Brownian motion.
    
    \begin{enumerate}
        \item We compute $dW_t^4$ and then write $W_t^4$ as the sum of an ordinary (Lebesque) integral with respect to time and a stochastic integral:
        \begin{align*}
            d(W_t^4) &= 4W_t^3 dW_t + \frac{1}{2}12W_t^2 \langle dW_t,dW_t \rangle\\
            &= 6W_t^2 dt + 4W_t^3 dW_t.
        \end{align*}
        In integral form:
        \begin{equation*}
            W_t^4 = \int_0^t 6W_u^2 du + \int_0^t 4W_u^3 dW_u.
        \end{equation*}
        
        \item Now take expectations on both sides of the formula you obtained in (a):
        \begin{align*}
            E(W_t^4) &= E\left(\int_0^t 6W_u^2 du\right) + E\left(\int_0^t 4W_u^3 dW_u\right)\\
            &= 6\int_0^t E(W_u^2) du\\
            &= 6 \int_0^t u du\\
            &= 6 \frac{u^2}{2}\bigg\vert_0^t\\
            &= 6\frac{t^2}{2}=3t^2,
        \end{align*}
        using the fact that $E(W_t^2)=t$. Therefore, $E(W_T^4)=3T^2$.
        
        \item We now use the method of (a) and (b) to derive a formula for $E(W_T^6)$:
        \begin{align*}
            d(W_t^6) &= 6W_t^5 dW_t + \frac{1}{2}30W_t^4 \langle dW_t,dW_t \rangle\\
            &= 15W_t^4 dt + 6W_t^5 dW_t.
        \end{align*}
        In integral form:
        \begin{equation*}
            W_t^6=\int_0^t 15W_u^4 du + \int_0^t 6W_u^5 dW_u.
        \end{equation*}
        Taking expectations on both sides,
        \begin{align*}
            E(W_t^6) &= 15\int_0^t E(W_u^4) du\\
            &= 45 \frac{u^3}{3}\bigg\vert^t_0\\
            &= 15t^3.
        \end{align*}
        Therefore, $E(W_T^6)=15T^3$.
    \end{enumerate}
    
    \vspace*{1.5cm}
    
    \item Let $(W_t)_{t\geq 0}$ be a Brownian motion and let $(\mathcal{G})_{t\geq 0}$ be an associated filtration, and let $\alpha_t$ and $\sigma_t$ be adapted processes. Define the Itô process
    \begin{equation}
        X_t = \int_0^t \sigma_s dW_s + \int_0^t \left(\alpha_s -\frac{1}{2}\sigma_s^2 \right) ds.
    \end{equation}
    
    \begin{enumerate}
        \item Write(1) in differential form.
        \begin{equation*}
            dX_t = \left(\alpha_t -\frac{1}{2}\sigma_t^2\right) dt + \sigma_t dW_t.
        \end{equation*}
    \end{enumerate}
    
    Consider an asset price process given by:
    \begin{equation}
        S_t=S_0e^{X_t},
    \end{equation}
    where $S_0$ is nonrandom and positive.
    
    \newpage
    
    \begin{enumerate}
        \setcounter{enumii}{1}
    
        \item Compute the differential of the stock price, $dS_t$, using Itô's formula. \newline
        
        We can write $S_t=S_0 e^{X_t} = f(X_t)$, where $f'(X_t)=S_0e^{X_t}$ and $f''(X_t)=S_0e^{X_t}$. \newline
        
        Therefore, the differential of $S_t=f(X_t)$ is given by:
        \begin{align*}
            dS_t &= S_0e^{X_t} dX_t + \frac{1}{2}S_0e^{X_t} dX_t dX_t\\
            &= S_t\left[\left(\alpha -\frac{1}{2}\sigma_t^2 \right) dt + \sigma_t dW_t\right] + \frac{1}{2}S_t\sigma_t^2 dt\\
            &= \alpha_t S_t dt + \sigma_t S_t dW_t. 
        \end{align*}

        \item What are the instantaneous mean rate of return and volatility of the stock price? \newline
        
        The asset price $S_t$ has instantaneous mean rate of return $\alpha_t$ and volatility $\sigma_t$. Both are time varying and random here.
        
        \newpage
        
	\vspace*{2.0cm}
	
        \item If $\alpha$ and $\sigma$ are constant, what is the distribution of the stock price and how do we call the process $dS_t$? \newline 
        
        If $\alpha$ and $\sigma$ are constant, we have the usual geometric Brownian motion model, and the distribution of $S_t$ is log-normal since:
        \begin{equation*}
            S_t = S_0 \exp\left\{\sigma W_t + \left( \alpha -\frac{1}{2}\sigma^2 \right) t \right\},
        \end{equation*}
        and $W_t \sim N(0,t)$.
    \end{enumerate}

    \newpage
    
    \item Let $(W_t)_{t\geq 0}$ be a Brownian motion. The \textit{Vasicek model} for the interest rate process $R_t$ is:
    \begin{equation}
        dR_t=(\alpha-\beta R_t) dt + \sigma dW_t,
    \end{equation}
    where $\alpha, \beta$ and $\sigma$ are positive constants. The solution to the stochastic differential equation (3) can be determined in closed form and is:
    \begin{equation}
        R_t=e^{-\beta t} R_0 +\frac{\alpha}{\beta}(1-e^{-\beta t}) + \sigma e^{-\beta t} \int_0^t e^{\beta s} dW_s.
    \end{equation}
    
    \begin{enumerate}
        \item In order to verify that (4) is indeed a solution to the SDE (3), we compute the differential of the right-hand side of (4). \newline 
        
        To do this, we use Itô's formula with
        \begin{equation*}
            f(t,x) = e^{-\beta t}R_0 + \frac{\alpha}{\beta}(1-e^{-\beta t}) + \sigma e^{-\beta t}x
        \end{equation*}
        and $X_t=\int_0^t e^{\beta s} dW_s$. Then the right-hand side of (4) is $f(t,X_t)$. For the Itô formula we need the following partial derivatives of $f(t,x)$:
        \begin{align*}
            f_t(t,x) &= -\beta e^{-\beta t}R_0 + \alpha e^{-\beta t} - \sigma\beta e^{-\beta t}x=\alpha -\beta f(t,x),\\
            f_x(t,x) &= \sigma e^{-\beta t},\\
            f_{xx}(t,x) &= 0.
        \end{align*}
        
        \vspace*{1cm}
        
        We also need the differential of $X_t$, which is $dX_t = e^{\beta t} dW_t$. Itô's formula then gives:
        \begin{align*}
            df(t,X_t) &= f_t(t,X_t) dt + f_x(t,X_t) dX_t + \frac{1}{2}f_{xx}(t,X_t) dX_t dX_t\\
            &= (\alpha-\beta f(t,x)) dt + \sigma dW_t.
        \end{align*}
        This shows that $f(t,X_t)$ satisfies the SDE (3) that defines $R_t$. \newline
        
        Moreover, $f(0,X_0)=R_0$. \newline
        
        Since $f(t,X_t)$ satisfies the equation defining $R_t$ and has the same initial condition, it must be the case that $f(t,X_t) = R_t$ for all $t\geq 0$. \newline
        
        \newpage 
        
        
        \item Use expression (4) to determine the distribution of the interest rate in the Vasicek model, and give explicitly its expectation and variance.\newline
        
        A stochastic integral of a deterministic function of time is normally distributed. \newline
        
        Thus, the random variable $\int_0^t e^{\beta s} dW_s$ appearing on the right-hand side of (4) is normal, with mean zero (remember that a stochastic integral is a martingale with initial value zero) and variance
        \begin{equation*}
            \int_0^t e^{2\beta s} ds = \frac{1}{2\beta}(e^{2\beta t}-1),
        \end{equation*}
        using Itô isometry. Therefore, $R_t$ is normally distributed with mean
        \begin{equation*}
            E(R_t) = e^{-\beta t}R_0 + \frac{\alpha}{\beta}(1-e^{-\beta t}),
        \end{equation*}
        and variance
        \begin{equation*}
            V(R_t) = \frac{\sigma^2}{2\beta}(1-e^{-2\beta t}).
        \end{equation*}
        
        \vspace*{2cm}
        
        \item Do you think the distribution of $R_t$ in the Vasicek model is appropriate for interest rates?\newline        
        
        Since $R_t$ is normally distributed, there is a positive probability that $R_t$ is negative, no matter how the parameters $\alpha, \beta$ and $\sigma$ are chosen. \newline
        
        This is an undesirable property for an interest rate model. \newline
        
        \newpage 
        
        \vspace*{1.5cm}
        
        \item The Vasicek model has the desirable property that the interest rate is \textit{mean-reverting}. Intuitively, what does this mean? \newline
        
        When $R_t = \frac{\alpha}{\beta}$, the drift term (the $dt$ term) in (3) is zero. \newline 
        
        When $R_t > \frac{\alpha}{\beta}$, the drift is negative, which pushes $R_t$ back towards $\frac{\alpha}{\beta}$. \newline
        
        When $R_t < \frac{\alpha}{\beta}$, the drift is positive, which again pushes $R_t$ back toward $\frac{\alpha}{\beta}$. \newline
        
        If $R_0 = \frac{\alpha}{\beta}$, then $E(R_t) = \frac{\alpha}{\beta}$ for all $t\geq 0$. If $R_0 \neq \frac{\alpha}{\beta}$, then \newline
        
        $\lim_{t\rightarrow \infty} E(R_t) = \frac{\alpha}{\beta}$, so $\frac{\alpha}{\beta}$ can 	be interpreted as the long-term mean of $R_t$.
    \end{enumerate}
    
    \newpage
    
    \item The time $t$ value of a European call option with maturity $T$ and strike price $K$ in the Black-Scholes model is given by:
    \begin{equation}
        c(t,S_t) = S_t \mathcal{N}(d_1) - K \exp(-r(T-t))\mathcal{N}(d_2),
    \end{equation}
    where
    \begin{equation*}
        d_1=\frac{\ln \frac{S_t}{K} + \left(r+\frac{\sigma^2}{2}\right)(T-t)}{\sigma\sqrt{T-t}},~~~~ d_2=d_1 - \sigma \sqrt{T-t}.
    \end{equation*}
    The derivatives of the function $c(t,S_t)$ with respect to various variables are called the \textit{Greeks}.
    \begin{enumerate}
        \item Compute the \textit{delta} of the option, i.e. the derivative of $c(t,S_t)$ with respect to the underlying stock price $S_t$. \newline 
        
        What happens to the price of the call option when the stock option decreases?\newline
        
        Hint: Be careful, remember that $d_1$ and $d_2$ are also functions of $S_t$! \newline
        
        First remember that $\mathcal{N}(y)$ is the cumulative standard normal distribution,
        \begin{equation*}
            \mathcal{N}(y) = \frac{1}{\sqrt{2\pi}}\int_{-\infty}^y e^{-\frac{z^2}{2}} dz,
        \end{equation*}
        so that
        \begin{equation*}
            \mathcal{N}'(y) = \frac{1}{\sqrt{2\pi}} e^{-\frac{y^2}{2}}.
        \end{equation*}
        Now, we can compute the derivative of $c(t,S_t)$ with respect to the underlying stock price $S_t$:
        \begin{equation*}
            c_S(t,S_t) = \mathcal{N}(d_1) + S_t\mathcal{N}'(d_1)\frac{\partial d_1}{\partial S} - K \exp(-r(T-t))\mathcal{N}'(d_2) \frac{\partial d_2}{\partial S}.
        \end{equation*}
        Note that
        \begin{equation*}
            K \exp(-r(T-t))\mathcal{N}'(d_2) = S_t \mathcal{N}'(d_1),
        \end{equation*}
        since
        \begin{align*}
            &K \exp(-r(T-t))\mathcal{N}'(d_2) = K \exp(-r(T-t))\frac{1}{\sqrt{2\pi}} e^{-\frac{d^2_2}{2}}\\
            &= K \exp(-r(T-t))\frac{1}{\sqrt{2\pi}} e^{-\frac{(d_1-\sigma\sqrt{T-t})^2}{2}}\\
            &= K \exp(-r(T-t))e^{\sigma\sqrt{T-t}d_1}e^{-\frac{\sigma^2(T-t)}{2}}\mathcal{N}'(d_1)\\
            &= K \exp(-r(T-t))\frac{S_t}{K}e^{\left(r+\frac{\sigma^2}{2}\right)(T-t)}e^{-\frac{\sigma^2(T-t)}{2}}\mathcal{N}'(d_1)\\
            &= S_t \mathcal{N}'(d_1)
        \end{align*}
        
        \vspace*{1.5cm}
        
        Note also that
        \begin{equation*}
            \frac{\partial d_1}{\partial S} = \frac{\partial d_2}{\partial S}.
        \end{equation*}
        Therefore,
        \begin{equation*}
            c_S(t,S_t) = \mathcal{N}(d_1) + S_t\mathcal{N}'(d_1) \frac{\partial d_1}{\partial S} - S_t\mathcal{N}'(d_1) \frac{\partial d_1}{\partial S} = \mathcal{N}(d_1).
        \end{equation*}
        
        Thus the delta of the call option is $\mathcal{N}(d_1)$. \newline 
        
        This is always positive, so we expect the price of the call option to fall when the underlying stock price decreases.
        
        \newpage
                
        \vspace*{2cm}
        
         \item Show that


	\begin{equation}
            c_t(t,S_t) = -rKe^{-r(T-t)}N(d_2) - \frac{\sigma S_t}{2\sqrt{T-t}}N'(d_1).
        \end{equation}
        
        This is the \textit{theta} of the option. \newline
        
        What is the effect of the passage of time on the value of the call option? \newline
        
        \newpage
        
        The derivative of the stock price with respect to time is:
        \begin{align*}
            & c_t(t,S_t) = \\
            &S_t \mathcal{N}'(d_1) \frac{\partial d_1}{\partial t} - rK \exp(-r(T-t)) \mathcal{N}(d_2) - K \exp(-r(T-t))\mathcal{N}'(d_1) \frac{\partial d_2}{\partial t}
        \end{align*}
        Now note that $\frac{\partial d_2}{\partial t}=\frac{\partial d_1}{\partial t} + \frac{\sigma}{2\sqrt{T-t}}$, so that
        \begin{align*}
            &c_t(t,S_t) \\
            &= S_t\mathcal{N}'(d_1)\frac{\partial d_1}{\partial t} - rk \exp(-r(T-t)) \mathcal{N}(d_2) 
            - S_t\mathcal{N}'(d_1)\left[ \frac{\partial d_1}{\partial t} + \frac{\sigma}{2\sqrt{T-t}}\right]\\
            &= - rK \exp(-r(T-t))\mathcal{N}(d_2) - \frac{S_t\sigma}{2\sqrt{T-t}}\mathcal{N}'(d_1),
        \end{align*}
        as we wanted to show.\\
        
        Since $\mathcal{N}(y)$ and $\mathcal{N}'(y)$ are always positive, the theta of the call option, $c_t(t,S_t)$, is always negative. This means that the call option price 		
        decreases as time goes by (time to maturity decreases), all else equal.
        
        \newpage
        
        \vspace*{1cm}
        
        \item Compute the \textit{gamma} of the option, i.e. the second derivative of $c(t,S_t)$ with respect to the underlying stock price $S_t$. \newline
        
        How do you interpret a positive gamma? \newline
        
        The gamma is
        \begin{equation*}
            c_{SS}(t,S_t) = \mathcal{N}'(d_1)\frac{\partial d_1}{\partial S} =\mathcal{N}'(d_1) \frac{1}{\sigma S_t\sqrt{T-t}}.
        \end{equation*}
        Like delta, gamma is always positive. \newline
        
        This means that the call price increases when the delta of the option (the first derivative with respect to $S$) increases. \newline
        
        \newpage
        
        \vspace*{0.50cm}
        
        \item How could you hedge a long position in the option against movements in the underlying stock price? \newline
        
        What are the other variables that drive the value of an option? \newline
        
        To hedge a long position in a call option one should hold $-c_S(t,S_t)$ (short delta) shares of the 
        stock and invest $Ke^{-r(T-t)}N(d_2)$ in the money market account. \newline
        
        The delta of the portfolio (the sensitivity to changes in the stock price) is thus zero when we set up the portfolio in this way, but since the gamma of the call is positive, the delta changes when the stock price moves, so we have to rebalance the portfolio to keep the delta of the overall portfolio equal to zero as the stock price changes. \newline
        
        Obviously, this portfolio is a hedge only against movements in the stock price, assuming that all other factors driving the option price, namely time $t$, interest rate $r$ and volatility of the underlying $\sigma$, are constant.
    \end{enumerate}
    
    \newpage
    \vspace*{1.75cm}
    
    A \textit{forward contract} with delivery price $K$ obligates its holder to buy one share of the stock at expiration time $T$ in exchange for payment $K$. \newline
    
    The terminal payoff of the forward contract is thus $S_T - K$. \newline
    
    The \textit{forward price} of a stock is the value of $K$ that makes the value of the forward contract zero at time $t$. \newline
    
    We can show by a simple replicating argument that with a constant interest rate $r$, the forward price at time $t$ is $F_t=e^{r(T-t)}S_t$.
    
    \newpage
    
    \begin{enumerate}
        \setcounter{enumii}{4}
    
    \vspace*{0.50cm}
    
        \item Explain how you could replicate the payoff of a \textit{put option} using a forward contract and a call option, and then use this observation to derive the formula for the price of a put option in the Black-Scholes model. \newline
        
        Let us consider a European put, which pays off $(K-S_T)^+$ at maturity $T$. We can observe that for any number $x$, the equation
        \begin{equation*}
            x-K = (x-K)^+ - (K-x)^+
        \end{equation*}
        holds. Namely, the payoff of a long call and a short put is equivalent to the payoff of a long forward:
        \begin{equation*}
            f(T,S_T) = c(T,S_T)-p(T,S_T).
        \end{equation*}
        Since the value at maturity is the same, these values must agree at all future times:
        \begin{equation}
            f(t,S_t) = c(t,S_t) -p(t,S_t), ~~ 0\leq t \leq T.
        \end{equation}
        If this were not the case, one could at some time $t$ either sell or buy the portfolio that is long the forward, short the call and long the put, realizing an instant profit, and have no liability upon expiration of the contracts. \newline
        
        \newpage
        
The relationship (7) is called \textit{put-call parity}, and we can use it to derive the expression for the put price within the Black-Scholes model, starting from the expression for the call price in the model and the value of the forward. \newline

Remember that the value of the forward contract is zero initially (at time $t=0$), but changes as time moves forward, and is $f(t,S_t) = S_t - e^{rt}S_0$ at time $t$. Therefore, the formula for the price of a put option in the Black-Scholes model is:
        \begin{align*}
            p(t,S_t) &= c(t,S_t) - [S_t - e^{rt}S_0]\\
            &= S_t\mathcal{N}(d_1) -K e^{-r(T-t)}\mathcal{N}(d_2) - S_t + e^{rt}S_0\\
            &= S_t[\mathcal{N}(d_1)-1]-Ke^{-r(T-t)} \left[\mathcal{N}(d_2) - e^{rT}\frac{S_0}{K}\right]\\
            &= S_t[\mathcal{N}(d_1)-1]-Ke^{-r(T-t)} [\mathcal{N}(d_2) -1]\\
            &= Ke^{-r(T-t)}\mathcal{N}(-d_2) - S_t\mathcal{N}(-d_1).
        \end{align*}
        
    \end{enumerate}
    
    
    \item This exercise shows how to create two correlated stock prices starting from independent Brownian motions.\\
    Suppose
    \begin{align}
        \frac{dS_1(t)}{S_1(t)} &= \alpha_1 dt + \sigma_1 dW_1(t)\\
        \frac{dS_2(t)}{S_2(t)} &= \alpha_2 dt + \sigma_2 \left[ \rho dW_1(t) + \sqrt{1-\rho^2} dW_2(t)\right],
    \end{align}
    where $W_1(t)$ and $W_2(t)$ are \textit{independent} Brownian motions and $\sigma_1>0,\sigma_2>0$ and $-1\leq \rho \leq 1$ are constant. \newline
    
    \newline To analyze the second stock price process, we define
    
    \begin{equation*}
        dW_3(t) = \rho dW_1(t) + \sqrt{1-\rho^2} dW_2(t).
    \end{equation*}
    
    \begin{enumerate}
        \item Show that $W_3(t)$ is a continuous martingale with $W_3(0)=0$ and $dW_3(t) dW_3(t) = dt$. \newline
        
         This implies that $W_3(t)$ is also a Brownian motion.\newline
        
        Since $W_1(t)$ and $W_2(t)$ are continuous martingales, also $W_3(t) = \rho dW_1(t) + \sqrt{1-\rho^2} dW_2(t)$ is a continuous martingale, due to linearity of the conditional expectation operator. \newline
        
        Moreover, \newline 
        \begin{align*}
            &dW_3(t)dW_3(t) \\
            &= \left(\rho dW_1(t) + \sqrt{1-\rho^2} dW_2(t)\right) \left(\rho dW_1(t) + \sqrt{1-\rho^2} dW_2(t)\right)\\
            &= \rho^2 dW_1(t)dW_1(t) + 2\rho\sqrt{1-\rho^2} dW_1(t)dW_2(t) + (1-\rho^2) dW_2(t)dW_2(t)\\
            &= \rho^2 dt + (1-\rho^2) dt= dt.
        \end{align*}
        
        \item Compute the differential of $W_1(t) W_3(t)$, then write it in integral form and compute the expectation $E(W_1(t)W_3(t)$). \newline
        
        What is the correlation between $W_1(t)$ and $W_3(t)$? \newline 
        
        According to Itô's formula,
        \begin{align*}
            d(W_1(t)W_3(t)) &= W_1(t) dW_3(t) + W_3(t) dW_1(t) + dW_1(t)dW_3(t)\\
            &= W_1(t) dW_3(t) + W_3(t) dW_1(t) +\rho dt.
        \end{align*}
        Integrating we obtain
        \begin{equation*}
            W_1(t) W_3(t) = \int_0^t W_1(s) dW_3(s) + \int_0^t W_3(s) dW_1(s) + \rho t.
        \end{equation*}
        The stochastic integrals on the right-hand side have expectation zero, so the covariance of $W_1(t)$ and $W_3(t)$ is
        \begin{equation*}
            E(W_1(t)W_3(t)) = \rho t.
        \end{equation*}
        Because both $W_1(t)$ and $W_3(t)$ have standard deviation $\sqrt{t}$, the number $\rho$ is the correlation between $W_1(t)$  and $W_3(t)$.
    \end{enumerate}
    
    
    \item Let a stock price be a geometric Brownian motion
    \begin{equation*}
        dS_t= \alpha S_t dt + \sigma S_t dW_t,
    \end{equation*}
    and let $r$ denote the interest rate. We define the \textit{market price of risk} to be
    \begin{equation*}
        \theta = \frac{\alpha -r}{\sigma},
    \end{equation*}
    and the \textit{state price density process} to be
    \begin{equation*}
        \zeta_t = \exp\left\{-\theta W_t -\left(r + \frac{1}{2}\theta^2\right) t\right\}.
    \end{equation*}
    
    \begin{enumerate}
        \item Show that
        \begin{equation*}
            d\zeta_t = -\theta \zeta_t dW_t - r\zeta_t dt.
        \end{equation*}\\
        We use Itô's formula with $\zeta_t=f(t,W_t)$. \newline
        
        For Itô's formula we need the following partial derivatives of $f(t,W_t)$:
        
        \begin{align*}
            f_t(t,W_t) &= -\left(r+\frac{1}{2}\theta^2\right) f(t,W_t),\\
            f_W(t,W_t) &= -\theta f(t,W_t),\\
            f_{WW}(t,W_t) &= \theta^2 f(t,W_t).
        \end{align*}
        
        \newpage
        
        Itô's formula then gives:
        
        \begin{align*}
            df(t,W_t) &= f_t(t,X_t) dt + f_W(t,W_t) dW_t + \frac{1}{2}f_{WW}(t,W_t) dW_tdW_t\\
            &= - \left(r + \frac{1}{2}\theta^2\right) f(t,W_t) dt - \theta f(t,W_t) dW_t + \frac{1}{2}\theta^2 f(t,W_t) dt\\
            &= -rf(t,W_t) dt - \theta f(t,W_t) dW_t,
        \end{align*}
        
        that is
        
        \begin{equation*}
            d\zeta_t = -r\zeta_t dt -\theta\zeta_t dW_t,
        \end{equation*}
        
        as we wanted to show. \newline
        
        \item Let $X$ denote the value of an investor's portfolio when he uses a portfolio process $\Delta_t$. We have,
        \begin{equation*}
            dX_t= rX_t dt + \Delta_t(\alpha -r) S_t dt + \Delta_t \sigma S_t dW_t.
        \end{equation*}
        Show that $\zeta_t X_t$ is a martingale.\newline 
        
        Hint: show that the differential $d(\zeta_t X_t)$ has no $dt$ term, i.e. no \textit{drift}. \newline
        
        \begin{align*}
            d(\zeta_t X_t) &= \zeta_t dX_t + X_t d\zeta_t + dX_td\zeta_t\\
            &= \zeta_t[rX_t dt + \Delta_t(\alpha -r)S_t dt + \Delta_t\sigma S_t dW_t] + X_t [-r\zeta_t dt - \theta\zeta_t dW_t] -\theta \sigma \Delta_t S_t \zeta_t dt\\
            &= \zeta_t[\Delta_t\sigma\theta S_t dt + \Delta_t\sigma S_t dW_t] - X_t\theta\zeta_t dW_t -\theta\sigma\Delta_t S_t\zeta_t dt\\
            &= \zeta_t(\sigma\Delta_t S_t - \theta X_t) dW_t.
        \end{align*}
        $d(\zeta_t X_t)$ has no drift, thus $\zeta_tX_t$ is a martingale.
        
        \item Let $T>0$ be a fixed terminal time. Show that if an investor wants to begin with some initial capital $X_0$ and invest in order to have portfolio value $V_T$ at time 			$T$, where $V_T$ is a given $\mathcal{G}_T$-measurable random variable, then he must begin with initial capital
        \begin{equation*}
            X_0 = E[\zeta_T V_T].
        \end{equation*}
        In other words, the present value at time zero of the random payment $V_T$ at time $T$ is $E[\zeta_T V_T]$. \newline
        
        This justifies calling $\zeta_t$ the state price density process. \newline
        
        By part (b), we know that $\zeta_tX_t$ is a martingale. Therefore,
        \begin{equation*}
            X_0=\zeta_0X_0 =E[\zeta_T X_T] = E[\zeta_T V_T].
        \end{equation*}
        This can be seen as a version of risk-neutral pricing, only that the pricing is carried out under the actual probability measure and discounting is done using the state price density $\zeta$, which is actually also called stochastic discount factor.
    \end{enumerate}
    
\end{enumerate}




\end{frame}



\end{document}







