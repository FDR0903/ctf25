\documentclass[11pt,a4,table]{article}

\renewcommand{\footnotesize}{\fontsize{9pt}{11pt}\selectfont}

%%%%%%%%%%%%%%%%%%%%%%%%%%%%%%%%%%%%%%%%%%%%%%%%%%%%

% Page Dimensions
\usepackage[top=1.0in, bottom=1.0in, left= 0.80in, right=0.80in]{geometry}

%%%%%%%%%%%%%%%%%%%%%%%%%%%%%%%%%%%%%%%%%%%%%%%%%%%%

% Caption options
\usepackage[margin= 16pt , footnotesize ,  sc , labelfont=bf , labelsep=period , format=plain , justification=centerfirst]{caption}


%%%%%%%%%%%%%%%%%%%%%%%%%%%%%%%%%%%%%%%%%%%%%%%%%%%%
%load minimum set of other required packages
\usepackage{amsfonts}
\usepackage{amsmath}
\usepackage{amssymb}
\usepackage{amsthm}
\usepackage{array}
\usepackage{bbm}
\usepackage{booktabs,longtable}
\usepackage{eurosym}
\usepackage{floatrow}
\usepackage{graphicx}
\usepackage{hyperref}
\usepackage{harvard}
\usepackage[english]{babel}
\usepackage{multirow}
\usepackage{lscape}
\usepackage{pdflscape}
\usepackage{rotating}
\usepackage{subfig}
\usepackage{setspace}
\usepackage{titlesec}
\usepackage{verbatim}
\usepackage{xcolor}
\usepackage[T1]{fontenc}

\usepackage[utf8]{inputenc}


%%%%%%%%%%%%%%%%%%%%%%%%%%%%%%%%%%%%%%%%%%%%%%%%%%%%
%%%%%%%%%%%%%%%%%%%%%%%%%%%%%%%%%%%%%%%%%%%%%%%%%%%%%

% and tikz
\usepackage{tikz}
\usetikzlibrary{decorations.pathreplacing}
	\definecolor{cinnabar}{rgb}{0.89, 0.26, 0.2}
	\definecolor{ceruleanblue}{rgb}{0.16, 0.32, 0.75}

% and optimally spaces tables

\usepackage{tabularx, booktabs}
\newcolumntype{Y}{>{\centering\arraybackslash}X}


%%%%%%%%%%%%%%%%%%%%%%%%%%%%%%%%%%%%%%%%%%%%%%%%%%%%

%% Roman numerals command thing.
\makeatletter
\newcommand{\rmnum}[1]{\romannumeral #1}
\newcommand{\Rmnum}[1]{\expandafter\@slowromancap\romannumeral #1@}
\makeatother

%%%%%%%%%%%%%%%%%%%%%%%%%%%%%%%%%%%%%%%%%%%%%%%%%%%%

% fake section command
\newcommand{\fakesection}[1]{%
  \par\refstepcounter{section}% Increase section counter
  \sectionmark{#1}% Add section mark (header)
  \addcontentsline{toc}{section}{\protect\numberline{\thesection}#1}% Add section to ToC
  % Add more content here, if needed.
}


%%%%%%%%%%%%%%%%%%%%%%%%%%%%%%%%%%%%%%%%%%%%%%%%%%%%

%% ------------------------------------------
%% Journal of Finance conventions
%% ------------------------------------------

\renewcommand{\thesection}{\Roman{section}}
\renewcommand{\thesubsection}{\Alph{subsection}}
\renewcommand{\thetable}{\Roman{table}}
\renewcommand{\abstractname}{\textsc{ABSTRACT}}
\makeatletter
\renewcommand{\baselinestretch}{1.10}
\renewcommand{\topfraction}{0.99}
\renewcommand{\bottomfraction}{0.99}
\setcounter{topnumber}{2}
\setcounter{bottomnumber}{2}
\setcounter{totalnumber}{4}
\setcounter{dbltopnumber}{2}
\renewcommand{\dbltopfraction}{0.9}
\renewcommand{\textfraction}{0.07}
\renewcommand{\floatpagefraction}{0.9}
\renewcommand{\dblfloatpagefraction}{0.9}
\titleformat{\section}{\centering\large\bfseries}{\thesection.}{1em}{}
\titleformat{\subsection}{\normalsize\itshape}{\thesubsection.}{1em}{}
\titleformat{\subsubsection}{\normalsize\itshape}{\thesubsubsection.}{1em}{}


%%%%%%%%%%%%%%%%%%%%%%%%%%%%

% math shocks cuts, theorems, lemmas etc.
\newcommand{\inner}[2]{\left < #1 \big | #2 \right >}
\newcommand{\proj}[2]{\mbox{proj} \left ( #1 \big | #2 \right )}
\newcommand{\cov}[2]{\mbox{cov}\left (#1 ,#2 \right )}
\newcommand{\var}[1]{\mbox{var}[#1]}
\newcommand{\Gray}[1]{{\color{gray}#1}}
\newcommand{\Red}[1]{{\color{red}#1}}


\newtheorem{theorem}{Theorem}
\newtheorem{acknowledgement}{Acknowledgement}
\newtheorem{algorithm}{Algorithm}
\newtheorem{axiom}{Axiom}
\newtheorem{case}{Case}
\newtheorem{claim}{Claim}
\newtheorem{conclusion}{Conclusion}
\newtheorem{condition}{Condition}
\newtheorem{conjecture}{Conjecture}
\newtheorem{corollary}{Corollary}
\newtheorem{criterion}{Criterion}
\newtheorem{definition}{Definition}
\newtheorem{example}{Example}
\newtheorem{exercise}{Exercise}
\newtheorem{lemma}{Lemma}
\newtheorem{notation}{Notation}
\newtheorem{problem}{Problem}
\newtheorem{proposition}{Proposition}
\newtheorem{hypothesis}{Hypothesis}
\newtheorem{remark}{Remark}
\newtheorem{solution}{Solution}
\newtheorem{summary}{Summary}

%%%%%%%%%%%%%%%%%%%%%%%%%%%%%%%%%%%%%%%%%%%%%%%%%%%%

% and choose line spacing (comes from setspace)

%\singlespacing
%\onehalfspacing
%\doublespacing
%\setstretch{1.5}
%\setstretch{2}

% make quote right flushed.
\makeatletter{}
\g@addto@macro\quote\flushright
\makeatother


%%%%%%%%%%%%%%%%%%%%%%%%%%%%%%%%%%%%%%%%%%%%%%%%%%%%%%%%%%%%%%%%%%%%%%
%%%%%%%%%%%%%%%%%%%%%%%%%%%%%%%%%%%%%%%%%%%%%%%%%%%%%%%%%%%%%%%%%%%%%%


\begin{document}

\vspace*{-0.7in}

\begin{center}
 \textbf{Continuous Time Finance: Problem Set 3} \\
 \textit{Introduction to Stochastic Processes and Stochastic Calculus} \\
Ilaria Piatti 
\end{center} 


\begin{enumerate}
    \item Let $S_t=S_0\exp\{\sigma W_t+(\alpha - \frac{1}{2}\sigma^2)t\}$ be a geometric Brownian motion and let $p$ be a positive constant. Compute $d(S_t^p)$, the differential of $S_t$ raised to the power of $p$.
    
    
    \item Let $W_t$ be a Brownian motion.
    
    \begin{enumerate}
        \item Compute $dW_t^4$ and then write $W_t^4$ as the sum of an ordinary (Lebesque) integral with respect to time and a stochastic integral.
        
        \item Take expectations on both sides of the formula you obtained in (a), use the fact that $E(W_t^2)=t$, and derive the formula $E(W_T^4)=3T^2$.
        
        \item Use the method of (a) and (b) to derive a formula for $E(W_T^6)$.
    \end{enumerate}
    
    
    \item Let $(W_t)_{t\geq 0}$ be a Brownian motion and let $(\mathcal{G})_{t\geq 0}$ be an associated filtration, and let $\alpha_t$ and $\sigma_t$ be adapted processes. Define the Itô process
    \begin{equation}
        X_t = \int_0^t \sigma_s dW_s + \int_0^t \left(\alpha_s -\frac{1}{2}\sigma_s^2 \right) ds.
    \end{equation}
    
    \begin{enumerate}
        \item Write (1) in differential form.
    \end{enumerate}
    
    Consider an asset price process given by:
    \begin{equation}
        S_t=S_0e^{X_t},
    \end{equation}
    where $S_0$ is nonrandom and positive.
    \begin{enumerate}
        \setcounter{enumii}{1}
    
        \item Compute the differential of the stock price, $dS_t$, using Itô's formula.
    
        \item What are the instantaneous mean rate of return and volatility of the stock price?
        
        \item If $\alpha$ and $\sigma$ are constant, what is the distribution of the stock price and how do we call the process $dS_t$?
    \end{enumerate}

    
    \item Let $(W_t)_{t\geq 0}$ be a Brownian motion. The \textit{Vasicek model} for the interest rate process $R_t$ is:
    \begin{equation}
        dR_t=(\alpha-\beta R_t) dt + \sigma dW_t,
    \end{equation}
    where $\alpha, \beta$ and $\sigma$ are positive constants. The solution to the stochastic differential equation (3) can be determined in closed form and is:
    \begin{equation}
        R_t=e^{-\beta t} R_0 +\frac{\alpha}{\beta}(1-e^{-\beta t}) + \sigma e^{-\beta t} \int_0^t e^{\beta s} dW_s.
    \end{equation}
    
    \begin{enumerate}
        \item Verify that (4) is indeed a solution of the SDE (3).\\
              Hint: compute the differential of the RHS of (4).
        
        \item Use expression (4) to determine the distribution of the interest rate in the Vasicek model, and give explicitly its expectation and variance.
        
        \item Do you think the distribution of $R_t$ in the Vasicek model is appropriate for interest rates?
        
        \item The Vasicek model has the desirable property that the interest rate is \textit{mean-reverting}. Intuitively, what does this mean?
    \end{enumerate}
    
    
    \item The time $t$ value of a European call option with maturity $T$ and strike price $K$ in the Black-Scholes model is given by:
    \begin{equation}
        c(t,S_t) = S_t \mathcal{N}(d_1) - K \exp(-r(T-t))\mathcal{N}(d_2),
    \end{equation}
    where
    \begin{equation*}
        d_1=\frac{\ln \frac{S_t}{K} + \left(r+\frac{\sigma^2}{2}\right)(T-t)}{\sigma\sqrt{T-t}},~~~~ d_2=d_1 - \sigma \sqrt{T-t}.
    \end{equation*}
    The derivatives of the function $c(t,S_t)$ with respect to various variables are called the \textit{Greeks}.
    \begin{enumerate}
        \item Compute the \textit{delta} of the option, i.e. the derivative of $c(t,S_t)$ with respect to the underlying stock price $S_t$. What happens to the price of the call option when the stock option decreases?\\
        Hint: Be careful, remember that $d_1$ and $d_2$ are also functions of $S_t$!
        
        \item Show that
        \begin{equation}
            c_t(t,S_t) = -rKe^{-r(T-t)}N(d_2) - \frac{\sigma S_t}{2\sqrt{T-t}}N'(d_1).
        \end{equation}
        This is the \textit{theta} of the option. What is the effect of the passage of time on the value of the call option?
        
        \item Compute the \textit{gamma} of the option, i.e. the second derivative of $c(t,S_t)$ with respect to the underlying stock price $S_t$. How do you interpret a positive gamma?
        
        \item How could you hedge a long position in the option against movements in the underlying stock price? What are the other variables that drive the value of an option?
    \end{enumerate}
    
    A \textit{forward contract} with delivery price $K$ obligates its holder to buy one share of the stock at expiration time $T$ in exchange for payment $K$. The terminal payoff of the forward contract is thus $S_T - K$. The \textit{forward price} of a stock is the value of $K$ that makes the value of the forward contract zero at time $t$. We can show by a simple replicating argument that with a constant interest rate $r$, the forward price at time $t$ is $F_t=e^{r(T-t)}S_t$.
    \begin{enumerate}
        \setcounter{enumii}{4}
    
        \item Explain how you could replicate the payoff of a \textit{put option} using a forward contract and a call option, and then use this observation to derive the formula for the price of a put option in the Black-Scholes model.
    \end{enumerate}
    
    \item This exercise shows how to create two correlated stock prices starting from independent Brownian motions.\\
    Suppose
    \begin{align}
        \frac{dS_1(t)}{S_1(t)} &= \alpha_1 dt + \sigma_1 dW_1(t)\\
        \frac{dS_2(t)}{S_2(t)} &= \alpha_2 dt + \sigma_2 \left[ \rho dW_1(t) + \sqrt{1-\rho^2} dW_2(t)\right],
    \end{align}
    where $W_1(t)$ and $W_2(t)$ are \textit{independent} Brownian motions and $\sigma_1>0,\sigma_2>0$ and $-1\leq \rho \leq 1$ are constant. To analyze the second stock price process, we define
    \begin{equation*}
        dW_3(t) = \rho dW_1(t) + \sqrt{1-\rho^2} dW_2(t).
    \end{equation*}
    
    \begin{enumerate}
        \item Show that $W_3(t)$ is a continuous martingale with $W_3(0)=0$ and $dW_3(t) dW_3(t) = dt$. This implies that $W_3(t)$ is also a Brownian motion.
        
        \item Compute the differential of $W_1(t) W_3(t)$, then write it in integral form and compute the expectation $E(W_1(t)W_3(t)$. What is the correlation between $W_1(t)$ and $W_3(t)$?
    \end{enumerate}
    
    
    \item Let a stock price be a geometric Brownian motion
    \begin{equation*}
        dS_t= \alpha S_t dt + \sigma S_t dW_t,
    \end{equation*}
    and let $r$ denote the interest rate. We define the \textit{market price of risk} to be
    \begin{equation*}
        \theta = \frac{\alpha -r}{\sigma},
    \end{equation*}
    and the \textit{state price density process} to be
    \begin{equation*}
        \zeta_t = \exp\left\{-\theta W_t -\left(r + \frac{1}{2}\theta^2\right) t\right\}.
    \end{equation*}
    
    \begin{enumerate}
        \item Show that
        \begin{equation*}
            d\zeta_t = -\theta \zeta_t dW_t - r\zeta_t dt.
        \end{equation*}
        
        \item Let $X$ denote the value of an investor's portfolio when he uses a portfolio process $\Delta_t$. We have,
        \begin{equation*}
            dX_t= rX_t dt + \Delta_t(\alpha -r) S_t dt + \Delta_t \sigma S_t dW_t.
        \end{equation*}
        Show that $\zeta_t X_t$ is a martingale.\\
        Hint: show that the differential $d(\zeta_t X_t)$ has no $dt$ term, i.e. no \textit{drift}.
        
        \item Let $T>0$ be a fixed terminal time. Show that if an investor wants to begin with some initial capital $X_0$ and invest in order to have portfolio value $V_T$ at time $T$, where $V_T$ is a given $\mathcal{G}_T$-measurable random variable, then he must begin with initial capital
        \begin{equation*}
            X_0 = E[\zeta_T V_T].
        \end{equation*}
        In other words, the present value at time zero of the random payment $V_T$ at time $T$ is $E[\zeta_T V_T]$. This justifies calling $\zeta_t$ the state price density process.
    \end{enumerate}
    
\end{enumerate}


\end{document}







