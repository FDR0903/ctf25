\documentclass[11pt,a4,table]{article}

\renewcommand{\footnotesize}{\fontsize{9pt}{11pt}\selectfont}

%%%%%%%%%%%%%%%%%%%%%%%%%%%%%%%%%%%%%%%%%%%%%%%%%%%%

% Page Dimensions
\usepackage[top=1.0in, bottom=1.0in, left= 0.80in, right=0.80in]{geometry}

%%%%%%%%%%%%%%%%%%%%%%%%%%%%%%%%%%%%%%%%%%%%%%%%%%%%

% Caption options
\usepackage[margin= 16pt , footnotesize ,  sc , labelfont=bf , labelsep=period , format=plain , justification=centerfirst]{caption}


%%%%%%%%%%%%%%%%%%%%%%%%%%%%%%%%%%%%%%%%%%%%%%%%%%%%
%load minimum set of other required packages
\usepackage{amsfonts}
\usepackage{amsmath}
\usepackage{amssymb}
\usepackage{amsthm}
\usepackage{array}
\usepackage{bbm}
\usepackage{booktabs,longtable}
\usepackage{eurosym}
\usepackage{floatrow}
\usepackage{graphicx}
\usepackage{hyperref}
\usepackage{harvard}
\usepackage[english]{babel}
\usepackage{multirow}
\usepackage{lscape}
\usepackage{pdflscape}
\usepackage{rotating}
\usepackage{subfig}
\usepackage{setspace}
\usepackage{titlesec}
\usepackage{verbatim}
\usepackage{xcolor}
\usepackage[T1]{fontenc}

\usepackage[utf8]{inputenc}


%%%%%%%%%%%%%%%%%%%%%%%%%%%%%%%%%%%%%%%%%%%%%%%%%%%%
%%%%%%%%%%%%%%%%%%%%%%%%%%%%%%%%%%%%%%%%%%%%%%%%%%%%%

% and tikz
\usepackage{tikz}
\usetikzlibrary{decorations.pathreplacing}
	\definecolor{cinnabar}{rgb}{0.89, 0.26, 0.2}
	\definecolor{ceruleanblue}{rgb}{0.16, 0.32, 0.75}

% and optimally spaces tables

\usepackage{tabularx, booktabs}
\newcolumntype{Y}{>{\centering\arraybackslash}X}


%%%%%%%%%%%%%%%%%%%%%%%%%%%%%%%%%%%%%%%%%%%%%%%%%%%%

%% Roman numerals command thing.
\makeatletter
\newcommand{\rmnum}[1]{\romannumeral #1}
\newcommand{\Rmnum}[1]{\expandafter\@slowromancap\romannumeral #1@}
\makeatother

%%%%%%%%%%%%%%%%%%%%%%%%%%%%%%%%%%%%%%%%%%%%%%%%%%%%

% fake section command
\newcommand{\fakesection}[1]{%
  \par\refstepcounter{section}% Increase section counter
  \sectionmark{#1}% Add section mark (header)
  \addcontentsline{toc}{section}{\protect\numberline{\thesection}#1}% Add section to ToC
  % Add more content here, if needed.
}


%%%%%%%%%%%%%%%%%%%%%%%%%%%%%%%%%%%%%%%%%%%%%%%%%%%%

%% ------------------------------------------
%% Journal of Finance conventions
%% ------------------------------------------

\renewcommand{\thesection}{\Roman{section}}
\renewcommand{\thesubsection}{\Alph{subsection}}
\renewcommand{\thetable}{\Roman{table}}
\renewcommand{\abstractname}{\textsc{ABSTRACT}}
\makeatletter
\renewcommand{\baselinestretch}{1.10}
\renewcommand{\topfraction}{0.99}
\renewcommand{\bottomfraction}{0.99}
\setcounter{topnumber}{2}
\setcounter{bottomnumber}{2}
\setcounter{totalnumber}{4}
\setcounter{dbltopnumber}{2}
\renewcommand{\dbltopfraction}{0.9}
\renewcommand{\textfraction}{0.07}
\renewcommand{\floatpagefraction}{0.9}
\renewcommand{\dblfloatpagefraction}{0.9}
\titleformat{\section}{\centering\large\bfseries}{\thesection.}{1em}{}
\titleformat{\subsection}{\normalsize\itshape}{\thesubsection.}{1em}{}
\titleformat{\subsubsection}{\normalsize\itshape}{\thesubsubsection.}{1em}{}


%%%%%%%%%%%%%%%%%%%%%%%%%%%%

% math shocks cuts, theorems, lemmas etc.
\newcommand{\inner}[2]{\left < #1 \big | #2 \right >}
\newcommand{\proj}[2]{\mbox{proj} \left ( #1 \big | #2 \right )}
\newcommand{\cov}[2]{\mbox{cov}\left (#1 ,#2 \right )}
\newcommand{\var}[1]{\mbox{var}[#1]}
\newcommand{\Gray}[1]{{\color{gray}#1}}
\newcommand{\Red}[1]{{\color{red}#1}}


\newtheorem{theorem}{Theorem}
\newtheorem{acknowledgement}{Acknowledgement}
\newtheorem{algorithm}{Algorithm}
\newtheorem{axiom}{Axiom}
\newtheorem{case}{Case}
\newtheorem{claim}{Claim}
\newtheorem{conclusion}{Conclusion}
\newtheorem{condition}{Condition}
\newtheorem{conjecture}{Conjecture}
\newtheorem{corollary}{Corollary}
\newtheorem{criterion}{Criterion}
\newtheorem{definition}{Definition}
\newtheorem{example}{Example}
\newtheorem{exercise}{Exercise}
\newtheorem{lemma}{Lemma}
\newtheorem{notation}{Notation}
\newtheorem{problem}{Problem}
\newtheorem{proposition}{Proposition}
\newtheorem{hypothesis}{Hypothesis}
\newtheorem{remark}{Remark}
\newtheorem{solution}{Solution}
\newtheorem{summary}{Summary}

%%%%%%%%%%%%%%%%%%%%%%%%%%%%%%%%%%%%%%%%%%%%%%%%%%%%

% and choose line spacing (comes from setspace)

%\singlespacing
%\onehalfspacing
%\doublespacing
%\setstretch{1.5}
%\setstretch{2}

% make quote right flushed.
\makeatletter{}
\g@addto@macro\quote\flushright
\makeatother


%%%%%%%%%%%%%%%%%%%%%%%%%%%%%%%%%%%%%%%%%%%%%%%%%%%%%%%%%%%%%%%%%%%%%%
%%%%%%%%%%%%%%%%%%%%%%%%%%%%%%%%%%%%%%%%%%%%%%%%%%%%%%%%%%%%%%%%%%%%%%


\begin{document}

\vspace*{-0.7in}

\begin{center}
 \textbf{Continuous Time Finance: Problem Set 2} \\
\textit{Pricing Principles and the Absence of Arbitrage} \\
Ilaria Piatti \\
\end{center} 


\begin{enumerate}
    \item Consider a two periods binomial model with parameters $r=0.25, u=1/d=2$ and $S_0=4$:
    
    \begin{enumerate}
        \item Compute the price of a European call option with strike price $K=5$ and maturity $T=2$.
    \end{enumerate}
    Consider now a European lookback option with a payoff at maturity $V_T$ given by:
    \begin{equation*}
        V_T=\max_{k=0,1,2}(S_k-K)^+,
    \end{equation*}
    where $K=5$, as in point (a) above.
    
    \begin{enumerate}
        \setcounter{enumii}{1}
    
        \item Compute for every trajectory of the binomial tree the corresponding value of $V_T$. Compare the results with the standard case of a European call option. Do you see any differences?
    
        \item Compute the quantities $\Delta_1$ and $\Delta_0$ which correspond to the \textit{delta} hedging factors for the lookback option.
        
        \item By means of the risk-neutral valuation principle determine the price at $t=0$ of the lookback option.
    \end{enumerate}
    
    
    \item Consider a \textit{n} periods binomial model with parameters $r=0.25, u=1/d=2$ and $S_0=4$. An European (American) put option with maturity $T$ and exercise price $K$ is the right, but not the obligation, to sell at date $T$ (or at one date $t\in(0,...,T)$) the underlying stock for a price equal to $K$.
    
    \begin{enumerate}
        \item Explain why the payoff $p_T$ at maturity of a European put option is given by
        \begin{equation*}
            p_T=(K-S_T)^+.
        \end{equation*}
        
        \item Compute the price $p_0$ at $t=0$ of a European put with parameters $T=1$ and $K=5$ in the binomial model.
        
        \item Compute the price $P_0$ of an American put option in a binomial model with parameters $T=1$ and $K=5$. First motivate the following equality:
        \begin{equation*}
            P_0=\max\{p_0,(K-S_0)^+\}.
        \end{equation*}
        
        \item Compute the price $p_0$ of a European put option in a binomial model with parameters $T=2$ and $K=5$ by first determining the corresponding hedging strategy.
        
        \item Compute the price $P_0$ of an American put option in a binomial model with parameters $T=2$ and $K=5$ by first determining the corresponding hedging strategy.
        \\\\
        Hint: First determine $P_1(H)$ and $P_1(T)$ using the idea of point (c). In a second step compute the value $V_0$ at date $t=0$ of a derivative instrument paying $P_1$ at date $t=1$. Finally determine $P_0$ as
        \begin{equation*}
            P_0 =\max\{V_0,(K-S_0)^+\}.
        \end{equation*}
    \end{enumerate}
    
    
    \item Consider a general \textit{n} periods binomial model and let us denote with $(\mathcal{G}_t)_{t=0,...,n}$ the filtration generated by the history of prices, $(S_t)_{t=0,...,n}$, so that $S_t$ is $\mathcal{G}_t$-measurable for any $t$.
    
    \begin{enumerate}
        \item Show that $E(S_t|\mathcal{G}_{t-1})=S_{t-1}(pu+(1-p)d).$
        
        \item Using the result in (a) and the tower property of conditional expectations, compute $E(S_{t+k}|\mathcal{G}_{t-1})$.
        
        \item Find the condition on the binomial probability $p$ such that the stock price process $(S_t)_{t=0,...,n}$ is a submartingale.
        
        \item Find the binomial probability $p$ such that the discounted price process $(S_t/B_t)_{t=0,...,n}$ is a martingale.
    \end{enumerate}
    
    
    \item Consider a self-financed portfolio process $(\Delta_t,\mathcal{G}_t)_{t=0,...,n}$ with value process $(X_t)_{t=,0,...,n}$, i.e.:
    \begin{equation*}
        X_{t+1} = \Delta_tS_{t+1} + (X_t-\Delta_t S_t)(1+r).
    \end{equation*}
    
    \begin{enumerate}
        \item Show that the value process $(X_t,\mathcal{G}_t)_{t=0,...,n}$ of the self-financed portfolio is adapted.
        
        \item Show that the discounted portfolio value process $(X_t/B_t,\mathcal{G}_t)_{t=0,..,n}$ is a martingale under the risk-neutral probability $\Tilde{P}$. Use the fact that the discounted stock price $S_t/B_t$ is a martingale under $\Tilde{P}$.
    \end{enumerate}
    
\end{enumerate}


\end{document}







