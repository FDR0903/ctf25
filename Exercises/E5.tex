\documentclass[11pt,a4,table]{article}

\renewcommand{\footnotesize}{\fontsize{9pt}{11pt}\selectfont}

%%%%%%%%%%%%%%%%%%%%%%%%%%%%%%%%%%%%%%%%%%%%%%%%%%%%

% Page Dimensions
\usepackage[top=1.0in, bottom=1.0in, left= 0.80in, right=0.80in]{geometry}

%%%%%%%%%%%%%%%%%%%%%%%%%%%%%%%%%%%%%%%%%%%%%%%%%%%%

% Caption options
\usepackage[margin= 16pt , footnotesize ,  sc , labelfont=bf , labelsep=period , format=plain , justification=centerfirst]{caption}


%%%%%%%%%%%%%%%%%%%%%%%%%%%%%%%%%%%%%%%%%%%%%%%%%%%%
%load minimum set of other required packages
\usepackage{amsfonts}
\usepackage{amsmath}
\usepackage{amssymb}
\usepackage{amsthm}
\usepackage{array}
\usepackage{bbm}
\usepackage{booktabs,longtable}
\usepackage{eurosym}
\usepackage{floatrow}
\usepackage{graphicx}
\usepackage{hyperref}
\usepackage{harvard}
\usepackage[english]{babel}
\usepackage{multirow}
\usepackage{lscape}
\usepackage{pdflscape}
\usepackage{rotating}
\usepackage{subfig}
\usepackage{setspace}
\usepackage{titlesec}
\usepackage{verbatim}
\usepackage{xcolor}
\usepackage[T1]{fontenc}

\usepackage[utf8]{inputenc}


%%%%%%%%%%%%%%%%%%%%%%%%%%%%%%%%%%%%%%%%%%%%%%%%%%%%
%%%%%%%%%%%%%%%%%%%%%%%%%%%%%%%%%%%%%%%%%%%%%%%%%%%%%

% and tikz
\usepackage{tikz}
\usetikzlibrary{decorations.pathreplacing}
	\definecolor{cinnabar}{rgb}{0.89, 0.26, 0.2}
	\definecolor{ceruleanblue}{rgb}{0.16, 0.32, 0.75}

% and optimally spaces tables

\usepackage{tabularx, booktabs}
\newcolumntype{Y}{>{\centering\arraybackslash}X}


%%%%%%%%%%%%%%%%%%%%%%%%%%%%%%%%%%%%%%%%%%%%%%%%%%%%

%% Roman numerals command thing.
\makeatletter
\newcommand{\rmnum}[1]{\romannumeral #1}
\newcommand{\Rmnum}[1]{\expandafter\@slowromancap\romannumeral #1@}
\makeatother

%%%%%%%%%%%%%%%%%%%%%%%%%%%%%%%%%%%%%%%%%%%%%%%%%%%%

% fake section command
\newcommand{\fakesection}[1]{%
  \par\refstepcounter{section}% Increase section counter
  \sectionmark{#1}% Add section mark (header)
  \addcontentsline{toc}{section}{\protect\numberline{\thesection}#1}% Add section to ToC
  % Add more content here, if needed.
}


%%%%%%%%%%%%%%%%%%%%%%%%%%%%%%%%%%%%%%%%%%%%%%%%%%%%

%% ------------------------------------------
%% Journal of Finance conventions
%% ------------------------------------------

\renewcommand{\thesection}{\Roman{section}}
\renewcommand{\thesubsection}{\Alph{subsection}}
\renewcommand{\thetable}{\Roman{table}}
\renewcommand{\abstractname}{\textsc{ABSTRACT}}
\makeatletter
\renewcommand{\baselinestretch}{1.10}
\renewcommand{\topfraction}{0.99}
\renewcommand{\bottomfraction}{0.99}
\setcounter{topnumber}{2}
\setcounter{bottomnumber}{2}
\setcounter{totalnumber}{4}
\setcounter{dbltopnumber}{2}
\renewcommand{\dbltopfraction}{0.9}
\renewcommand{\textfraction}{0.07}
\renewcommand{\floatpagefraction}{0.9}
\renewcommand{\dblfloatpagefraction}{0.9}
\titleformat{\section}{\centering\large\bfseries}{\thesection.}{1em}{}
\titleformat{\subsection}{\normalsize\itshape}{\thesubsection.}{1em}{}
\titleformat{\subsubsection}{\normalsize\itshape}{\thesubsubsection.}{1em}{}


%%%%%%%%%%%%%%%%%%%%%%%%%%%%

% math shocks cuts, theorems, lemmas etc.
\newcommand{\inner}[2]{\left < #1 \big | #2 \right >}
\newcommand{\proj}[2]{\mbox{proj} \left ( #1 \big | #2 \right )}
\newcommand{\cov}[2]{\mbox{cov}\left (#1 ,#2 \right )}
\newcommand{\var}[1]{\mbox{var}[#1]}
\newcommand{\Gray}[1]{{\color{gray}#1}}
\newcommand{\Red}[1]{{\color{red}#1}}


\newtheorem{theorem}{Theorem}
\newtheorem{acknowledgement}{Acknowledgement}
\newtheorem{algorithm}{Algorithm}
\newtheorem{axiom}{Axiom}
\newtheorem{case}{Case}
\newtheorem{claim}{Claim}
\newtheorem{conclusion}{Conclusion}
\newtheorem{condition}{Condition}
\newtheorem{conjecture}{Conjecture}
\newtheorem{corollary}{Corollary}
\newtheorem{criterion}{Criterion}
\newtheorem{definition}{Definition}
\newtheorem{example}{Example}
\newtheorem{exercise}{Exercise}
\newtheorem{lemma}{Lemma}
\newtheorem{notation}{Notation}
\newtheorem{problem}{Problem}
\newtheorem{proposition}{Proposition}
\newtheorem{hypothesis}{Hypothesis}
\newtheorem{remark}{Remark}
\newtheorem{solution}{Solution}
\newtheorem{summary}{Summary}

%%%%%%%%%%%%%%%%%%%%%%%%%%%%%%%%%%%%%%%%%%%%%%%%%%%%

% and choose line spacing (comes from setspace)

%\singlespacing
%\onehalfspacing
%\doublespacing
%\setstretch{1.5}
%\setstretch{2}

% make quote right flushed.
\makeatletter{}
\g@addto@macro\quote\flushright
\makeatother


%%%%%%%%%%%%%%%%%%%%%%%%%%%%%%%%%%%%%%%%%%%%%%%%%%%%%%%%%%%%%%%%%%%%%%
%%%%%%%%%%%%%%%%%%%%%%%%%%%%%%%%%%%%%%%%%%%%%%%%%%%%%%%%%%%%%%%%%%%%%%


\begin{document}

\vspace*{-0.7in}

\begin{center}
 \textbf{Continuous Time Finance: Problem Set 5} \\
 \textit{Change of Num\'{e}raire}		\\
Ilaria Piatti
\end{center} 


\begin{enumerate}
    \item \textbf{Foreign and Domestic Risk-Neutral Measures}\\
    Suppose we have a market with two currencies, which we call foreign and domestic.\\
    The model is driven by a two-dimensional Brownian motion
    \begin{equation*}
        W(t) = (W_1(t), W_2(t))
    \end{equation*}
    and we assume that $W_1$ and $W_2$ are independent under $P$.\\
    We begin with a stock whose price $S(t)$ in domestic currency satisfies
    \begin{equation}
        dS(t) = \alpha(t) S(t) dt + \sigma_1(t) S(t) dW_1(t).
    \end{equation}
    The domestic interest rate is $R(t)$, so that the domestic money market account price and domestic discount process are:
    \begin{equation*}
        M(t) = e^{\int_0^t R(u) du}, \textit{ and } D(t) = e^{-\int_0^t R(u) du}.
    \end{equation*}
    There is also a foreign interest rate $R^f(t)$, which leads to the foreign money market account price and foreign discount process
    \begin{equation*}
        M^f(t)=e^{\int_0^t R^f(u) du}, \textit{ and } D^f(t) = e^{-\int_0^t R^f(u) du}.
    \end{equation*}
    Finally, there is an exchagne rate $Q(t)$, which gives units of domestic currency per unit of foreign currency. We assume it satisfies
    \begin{equation}
        dQ(t) = \gamma(t)Q(t) dt + \sigma_2(t) Q(t) dW_3(t),
    \end{equation}
    where $dW_3(t) = \rho(t) dW_1(t) + \sqrt{1-\rho^2(t)} dW_2(t)$. i.e. $W_3$ is a Brownian motion correlated with $W_1$ and $W_2$.
    
    \begin{enumerate}
        \item Compute the differential of the stock price in units of the domestic money market account, i.e. $d(D(t)S(t))$.
        
        \item Find the process $\Theta_1(t)$ such that
        \begin{equation*}
            d(D(t)S(t)) = \sigma_1(t)D(t)S(t) d\Tilde{W}_1(t),
        \end{equation*}
        by defining $\Tilde{W}_1(t)=\int_0^t\Theta_1(u) du + W_1(t)$.
        
        \item The value of the foreign money market account in domestic currency is $M^f(t)Q(t)$, and its discounted value is $D(t)M^f(t)Q(t)$. Find the differential of this price.\\
        Hint: Use the fact that $dM^f(t) = R^f(t) M^f(t) dt$, then compute $d(M^f(t)Q(t))$, and finally $d\left(D(t)M^f(t)Q(t)\right)$.
        
        \item Find the processes $\Theta_2(t)$ such that
        \begin{equation*}
            d\left(D(t)M^f(t)Q(t)\right) = D(t)M^f(t)Q(t) \left[\sigma_2(t)\rho(t) d\Tilde{W}_1(t) + \sigma_2(t)\sqrt{1-\rho^2(t)} d\Tilde{W}_2(t)\right],
        \end{equation*}
        by defining $\Tilde{W}_2(t) = \int_0^t \Theta_2(u) du + W_2(t)$.
    \end{enumerate}
    
    Since there is only one possible solution for $\Theta_1(t)$ and $\Theta_2(t)$, there is a unique risk-neutral measure $\Tilde{P}$. Under this measure, $\Tilde{W}(t)=\left(\Tilde{W}_1(t),\Tilde{W}_2(t)\right)$ is a two-dimensional Brownian motion and all the assets in units of the domestic money market account are martingales. Let us also define
    \begin{equation*}
        d\Tilde{W}_3(t)=\rho(t) d\Tilde{W}_1(t) + \sqrt{1-\rho^2(t)} d\Tilde{W}_2(t)
    \end{equation*}
    
    \begin{enumerate}
        \setcounter{enumii}{4}
    
        \item We can write the processes $D(t)S(t)$ and $D(t)M^f(t)Q(t)$ in undiscounted form by multiplying them by $M(t)=1/D(t)$ and using the formula $dM(t) = R(t)M(t) dt$ and Itô product rule. Show that both the stock price and foreign money market account in units of the domestic currency have instantaneous return $R(t)$ under the domestic risk-neutral measure. Also give explicitly the volatility vector of the foreign money market account in units of the domestic currency.
    \end{enumerate}
    
    Now, to find the foreign risk-neutral measure, we take the foreign money market account as numéraire. Its value denominated in units of the domestic currency is $M^f(t)Q(t)$.
    \begin{enumerate}
        \setcounter{enumii}{5}
        
        \item Define the risk-neutral measure under this new numéraire, $\Tilde{P}^f$, and show that the stock and the domestic money market account, both in units of the new numéraire, are martingales under $\Tilde{P}^f$.
    \end{enumerate}
    
    
    \item \textbf{Proof of Theorem 4}\\
    \\
    The goal of this exercise is to prove Theorem 4 in Lecture 7, which basically gives the equivalent of Black-Scholes call price formula but for the case of random interest rate. We prove it for $t=0$ but it is not difficult to modify the proof to account for general $t$.\\
    The formula is the following:
    \begin{equation}
        V(0) = S(0)\mathcal{N}(d_1) - K B (0,T) \mathcal{N}(d_2),
    \end{equation}
    with
    \begin{equation*}
        d_1=\frac{\ln \frac{For_S(0,T)}{K}+\frac{\sigma^2}{2}T}{\sigma\sqrt{T}}, d_2 = d_1 - \sigma\sqrt{T}.
    \end{equation*}
    
    \begin{enumerate}
        \item The formulah olds if the forward price of the underlying follows
        \begin{equation}
            dFor_S(t,T) = \sigma For_S(t,T) d\Tilde{W}^T(t),
        \end{equation}
        for a positive constant $\sigma$.\\
        Show that the solution of (6) (is it supposed to be 4 here?) is
        \begin{equation}
            For_S(t,T)=\frac{S(0)}{B(0,T)}e^{\sigma\Tilde{W}^T(t)-\frac{1}{2}\sigma^2t}.
        \end{equation}
        
        \item What is the distribution of the log forward price?
        
        \item We will need one more change of measure. Suppose we take the asset price $S(t)$ to be the numéraire. Define the risk-neutral measure for this numéraire.
        
        \item Denominated in units of $S(t)$, the zero-coupon bond is
        \begin{equation*}
            \frac{B(t,T)}{S(t)}=\frac{1}{For_S(t,T)}, 0\leq t \leq T.
        \end{equation*}
        Compute the differential of $\frac{1}{For_S(t,T)}$ and show that it is a martingale under $\Tilde{P}^S$.
        
        \item Show that the solution of the differential in point (d) is
        \begin{equation}
            \frac{1}{For_S(t,T)} = \frac{B(0,T)}{S(0)}e^{-\sigma\Tilde{W}^S(t)-\frac{1}{2}\sigma^2t},
        \end{equation}
        which has a log-normal distribution under $\Tilde{P}^S$, the measure under which $\Tilde{W}^S(t)$ is a Brownian motion.
    \end{enumerate}
    
    At time zero, the value of a European call expiring at time $T$, according to the risk-neutral pricing formula is
    \begin{align}
        V(0) &= \Tilde{E}[D(T)(S(T)-K)^+] \nonumber \\
        &= \Tilde{E}[D(T)S(T)\mathbbm{1}_{\{S(T)>K\}}] - K\Tilde{E}[D(T)\mathbbm{1}_{\{S(T)>K\}}] \nonumber \\
        &= S(0)\Tilde{E}\left[\frac{D(T)S(T)}{S(0)}\mathbbm{1}_{\{S(T)>K\}}\right] - KB(0,T)\Tilde{E}\left[\frac{D(T)}{B(0,T)}\mathbbm{1}_{\{S(T)>K\}}\right] \nonumber \\
        &= S(0)\Tilde{P}^S\{S(T)>K\} - KB(0,T)\Tilde{P}^T\{S(T)>K\} \nonumber \\
        &= S(0)\Tilde{P}^S\{For_S(T,T)>K\} - KB(0,T)\Tilde{P}^T\{For_S(T,T)>K\} \nonumber \\
        &= S(0)\Tilde{P}^S\left\{\frac{1}{For_S(T,T)}<\frac{1}{K}\right\} - KB(0,T)\Tilde{P}^T\{For_S(T,T)>K\}.
    \end{align}
    \begin{enumerate}
        \setcounter{enumii}{5}
        
        \item Use the fact that $\Tilde{W}^S(t)$ is normal to show that
        \begin{equation*}
            \Tilde{P}^S\left\{\frac{1}{For_S(T,T)} < \frac{1}{K}\right\} = \mathcal{N}(d_1).
        \end{equation*}
        
        \item Use the fact that $\Tilde{W}^T(t)$ is normal to show that
        \begin{equation*}
            \Tilde{P}^T\{For_S(T,T)>K\} = \mathcal{N}(d_2).
        \end{equation*}
    \end{enumerate}
    
\end{enumerate}


\end{document}







